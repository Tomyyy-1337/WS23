\documentclass[a4paper]{scrartcl}

\usepackage[utf8]{inputenc}
\usepackage[ngerman]{babel}

\usepackage{url,amsmath,amssymb,mathrsfs,enumerate,dsfont}
\usepackage[space,extendedchars]{grffile}
\usepackage{algorithm,algorithmic}
\usepackage{verbatim}
\usepackage{listings}
\usepackage{geometry}
\usepackage{tikz}
\usepackage{etoolbox}
\usetikzlibrary{automata,arrows}
\usepackage{fancyhdr}
\usepackage{subfigure}
\usepackage[ngerman]{babel}
\usepackage{hyperref}
\usepackage{blindtext}
\usepackage{framed}
\usepackage{paralist}
\usepackage{multirow}

\def\ojoin{\setbox0=\hbox{$\bowtie$}%
  \rule[-.02ex]{.25em}{.4pt}\llap{\rule[\ht0]{.25em}{.4pt}}}
\def\leftouterjoin{\mathbin{\ojoin\mkern-5.8mu\bowtie}}
\def\rightouterjoin{\mathbin{\bowtie\mkern-5.8mu\ojoin}}
\def\fullouterjoin{\mathbin{\ojoin\mkern-5.8mu\bowtie\mkern-5.8mu\ojoin}}

\usetikzlibrary{arrows,shapes, automata}
\setkomafont{disposition}{\normalfont\bfseries}
\setlength\parindent{0pt}

%%%%%%%%%%%%%%%%%%%%%%%%%
% BITTE HIER ANPASSEN.  %
\title{Web 2.0}
\author{Tom Paßberg , Iain Dorsch, }
%%%%%%%%%%%%%%%%%%%%%%%%%

\begin{document}
\maketitle
\section*{Aufgabe 2}
\begin{enumerate}[a)]
    \item Das Internet besteht aus allen Netzwerken, allen Computern und allen Daten. Das WWW ist eine
    von vielen Anwendungen des Internets. WWW bezeichnet die Menge aller Webseiten, die über das Internet 
    von Webservern abrufbar sind.
    \item RFC: Request for Comments
    \item DNS: Domain Name System
    \item \texttt{1101:0ABC:0000:0000:A002:0000:002F:10FF} $\to$ \texttt{1101:ABC::A002:0:2F:10FF}
    \item Wenn man mehrfach "::" verwenden dürfte würden mehrere Adressen die selbe komprimierte Form 
    teilen. Damit wäre es nicht möglich die vollständige IP eindeutig zu rekonstruieren. \\
    Beispiel: \\
    \texttt{B100:1ABC:0000:0000:002F:0000:0000:ABCD} und \\
    \texttt{B100:1ABC:0000:002F:0000:0000:0000:ABCD} würden beide \\
    \texttt{B100:1ABC::2F::ABCD} ergeben.
    \item Bei jeder Verbindung über das TCP-Protokoll werden zwei eindeutige Endpunkte
    identifiziert, die ein geordnetes Paar darstellen. Jedes Paar besteht aus einer IP-Adresse und einem Port.
    Die TCP-Verbindung wird also durch 4 Parameter beschrieben.
\end{enumerate}


\end{document}
